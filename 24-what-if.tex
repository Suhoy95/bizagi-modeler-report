\chapter{Анализ <<Что-Если>>}

Теперь попробуем промоделировать очевидное решение проблемы,
наблюдаемой нами в ходе всех этапов анализа: долгое время
обработки и подготовки сведений о судимости в информационном центре.

Для этого воспользуемся еще одной замечательной возможностью
Bizagi Modeler: \textit{What-If-Analysis} или Анализ Что-Если.
В процессе его проведения, мы создадем копию моделируемого
процесса (Рис. \ref{what-if}), после чего вносим в копию
модели правки, которые желаем провести в бизнес-процессе.
Далее программа сама может быстро провести оба моделирвоания,
и предоставить сравнительный отчет работы этих двух систем.

В нашем случае, представим, что информационный центр
модернизировали и сделали обработку требуемого запроса
автоматической с помощью сервиса, который способен выполнить
его в течении 5 минут, при этом возможна обработка до 100
одновременных запросов.

Запустив анализ мы получаем отчет (Рисунки
\ref{what-if-resources-report},
\ref{what-if-common-report},
\ref{what-if-inc-report} и
\ref{what-if-inc-report-2}). Как можем заметить, нагрузка на
ИЦ упала до 0.04\%, что вполне ожидаемо. Однако мы не
получили значительного повышения производительности в бизнес
процессе: да, максимальное время выполнение уменьшилось и
составляет 61-62 дня, что приемлемо для нашего регламента,
однако же среднее время выполнения увеличилось на целых два
дня.

В чем же такая значительная проблема? Просмотрев другие задачи,
можно увидеть, что из-за быстрой обработки запросов
информационным центром, большой поток заявок нагрузил других
исполнителей процесса, что вызвало задержки ожидания работников,
что в конечном счете повлияло на среднее время выполнение почти
что всех задач не в лучшую сторону. И таким образом,
выигрышь 30 дней задержки при обработке информационным центром
оказался не таким значительным, как ожидалось.

\myImage{Создание копии модели процесса}{what-if}{what-if}
\myImage{Сравнительный отчет ресурсов: нагрузка на ИЦ заментно снизилась
}{what-if-resources-report}{what-if-resources-report}
\myImage{Максимальное время выполнения уменьшилось до
допустимых пределов, чего не скажешь о среднем времени
выполнения запроса}{what-if-common-report}{what-if-common-report}
\myImage{Как видно среденее время выполнения других задач
существенно выросло}{what-if-inc-report}{what-if-inc-report}
\myImage{Виден выигрыш в 30 дней на обработке запроса в ИЦ,
как и увеличение среднего времени выполнения оставшихся
задач}{what-if-inc-report-2}{what-if-inc-report-2}


