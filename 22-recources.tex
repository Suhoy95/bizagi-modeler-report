\chapter{Анализ используемых ресурсов}

Следующим этапом нашей работы будет имитация исследуемого
процесса, с учететом зартрачиваемых ресрсов. Основными
ресурсами в нашем процессе являются исполнители, или
работники того или иного подразделения. Информационный
центр и ТКДНиЗП мы считаем за одну единую сущность и не
затрагиваем их внутреннюю структуру:

\begin{itemize}
    \item Работник МФЦ (5);
    \item Работник Министерства (3);
    \item Должностное лицо Министерства (1);
    \item ИЦ (15);
    \item ТКДНиЗП (1);
    \item Ксерокс (4);
    \item Наградной лист (1).
\end{itemize}

В данном списке стоит сделать уточнение, что 15 единиц
информационного центра в нашем понимании вовсе не означает,
что в реальности будет 15 информационных центров, просто
в рамках информационного центра может обрабатываться 15
параллельных процессов (запросов). Также в качестве ресурсов,
помимо исполнителей, требуются инструменты (ксерокс, для
снятия копий регистрируемых документов) и расходные
материалы (бланк наградного листа).

После того, как мы задали эти ресурсы в свойствах модели,
зададим  стоимость работы каждого ресурса: затраты будем
рассматривать со стороны министерства и МФЦ, поэтому
стоимость работ ИЦ и ТКДНиЗП оставим нулевым
(Рис. \ref{resources_costs}).

Запустив моделирование такого процесса, в отчете мы увидим,
как сильно был задействован тот или иной ресурс в нашей
системе (\ref{resources_report_1}).

\myImage{Стоимость работы описанных ресурсов
}{resources_costs}{resources_costs}
\myImage{ИЦ и ТКДНиЗП очень загружены
}{resources_report_1}{resources_report_1}
\clearpage

Как можно заметить, ИЦ и ТКДНиЗП очень активно задействованы
при обработке потока заявок. Проведя несколько опытов,
подберем количество ресурсов, чтобы нагрузка на исполнителей
при моделировании была более сбалансирована и более корректна
для реальной ситуации (Рис. \ref{resources_new} - новые ресурсы,
Рис. \ref{resources_report_new} - новый отчет).

После нормализации нагрузки на исполнителей, полезно
взглянуть, как ограничения по ресурсам сказались на времени
выполнения задач. Основным сдерживающим фактором, повившемся
на этом этапе является ожидание того или иного ресурса.
Отсортируем по убыванию среднее ожидание ресурса и
опять увидим, что основной причиной для задержки остается -
ИЦ (Рус. \ref{resources_report_new_time} и
\ref{resources_report_new_time_2}). Причем не смотря
на большое количество исполнителей, которое было добавлено
при нормализации модели, среднее время ожидания составило
целых 20 дней.

\clearpage
\myImage{Новое распределение ресурсов
}{resources_new}{resources_new}
\myImage{Нагрузка на исполнителей в районе 50\% -- оптимально
}{resources_report_new}{resources_report_new}

\myImage{Временной отчет результатов, среднее время
обработки запроса в ИЦ значительно выросло
}{resources_report_new_time}{resources_report_new_time}
\myImage{Продолжение временного отчета, время обработки запроса
в ИЦ увеличелось из-за ожидания исполнителя
}{resources_report_new_time_2}{resources_report_new_time_2}
