\chapter{Временной анализ}

Проверив составленную модель на валидность, перейдем
к следующему этапу анализа - временому. В этом анализе
мы опишем, сколько времени выполняется каждая подзадача
и оценим, как эффективно выполняется процесс с этими
ограничениями.

В таблице \ref{table:time} представлены временные затарты
на простые задачи, присутствующие в нашей схеме. Задать их
в Bizagi Modeler очень просто: достаточно нажать на значек
часов справа от задачи, после чего ввести требуемое время
(Рисунок \ref{time_set}).

\begin{table}
    \begin{tabular}{|l|c|}
        \hline
        Проверить и зарегистрировать документы & 15min \\ \hline
        Запросить сведения о наличии судимости & 30min \\ \hline
        Занести сведения о судимости в журнал & 10min \\ \hline
        Проверить соблюдение прав детей & 30min \\ \hline
        Подтвердить соблюдение прав детей & 6hours \\ \hline
        Подготовить наградной лист & 2hours \\ \hline
        Подготовить предложения о награждении & 2hours \\ \hline
    \end{tabular}
    \caption{Время выполнения подзадач в процессе}
    \label{table:time}
\end{table}

\myImage{Задание времени выполнения элементарной задачи
}{time_set}{time_set}
\clearpage

\myImage{Задание времени выполнения обработки запроса в ИЦ
с помощью нормального распределения}{time_randn}{time_randn}

В предыдущей таблице мы не рассмотрели один этап -
обработку запроса в ИЦ на наличие судимости о заявителя.
Так как это сложный процесс, и нам неизвестна внутренняя
структура обработки запроса, смоделируем его время выполнения
как нормальное распределение (Рисунок \ref{time_randn}).
Будем считать, что в среднем обработка запроса будет
4-ре дня (5760 минут) со стандартным отклонением в 1000 минут
(~0.7дня).

Запустив заполненую модель в режиме временого анализа,
мы получем новый, более детальный, отчет (Рисунок
\ref{time_report}). Тут больше всего нас интерисуют колонки
времени выполнения каждого этапа задачи (минимальное,
максимально и среднее на каждую задачу).

Отсортировав по среднему времени выполнения задачи,
можно отчетливо видить узкое горлышко нашего процесса -
среднее время выполнения почти полностью требуется на обработку
запроса в информационном центре. При этом, если повезет,
данная процедура может пройти за 10 часов, при условии, что
заявитель предоставил эти сведения о судимости
при регистрации документов.

\myImage{Отчет симуляции для анализа времени, затраченного
на бизнес-процесс}{time_report}{time_report}
