\chapter{Календарный анализ}

Последним уточнением в нашей модели станет \textit{календарный
анализ}. По сути это является уточнением временого анализа,
с учетом того, что обчно исполнители (люди) не могут
работать 24 часа в сути 7 дней в неделю. Поэтому в нашей
модели создадим стандартный календарь 8-ми часового рабочего
дня с пятидневной рабочей неделей (Рис. \ref{calendar}),
после чего назначим этот календарь для всех исполнителей
(Рис. \ref{calendar_res}).

Запустив модель, мы получим более уточненные данные модели:
что и следовало ожидать, время выполнения увеличилось, но
пряжние наблюдения остались актуальны: обращение в ИЦ занимает
больше всего времени в процессе. Более того, можно заметить,
что при неудачном стичении обстоятельств оказание услуги
может проходить чуть дольше, чем 65 дней -- а это между тем
нарушение установленного регламента. Исправление этой
проблемы мы выполним в следующим разделе.

\myImage{8 часовой рабочий день. 40 часов в неделю
}{calendar}{calendar}
\myImage{Все исполнители работают по рабочему календарю
}{calendar_res}{calendar_res}
\myImage{Отчет. Время процесса и обработка запроса в ИЦ
увеличелись на 5 дней }{calendar_report}{calendar_report}