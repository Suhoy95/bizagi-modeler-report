\Introduction

В лабораторной работе проводится знакомство с моделью оказания
государственной услуги \textit{<<Подготовка предложений о представлении
к награждению знаком отличия Свердловской области <<Совет да любовь>>(
Дфлее, услуга Совет да любовь)}. На основе этой услуги
изучаются вопросы имитационного моделирования с использованием
промышленного стандарта BPMN.


В лабораторной работе студенты знакомятся с моделью
бизнес-процесса оказания услуг скорой помощи. На его основе
изучаются  важные  вопросы
имитационного  моделирования:  составление  модели  с  использованием
промышленного   стандарта
BPMN
2.
0
,   использование   статистической
информации  для  моделирования,  анализ  результатов  моделирования  и
оптимизация   бизнес
-
проц
есса.   Работа   выполняется   с   использованием
программы
Bizagi
Process
Modeler
.
Данное  методическое  пособие  является  адаптированным  переводом
официального   руководства,   доступного   на   сайте   компании
Bizagi
http://help.bizagi
.com/

Целью работы является создание всякой всячины. Для достижения поставленной цели необходимо решить следующие задачи:

\begin{itemize}
\item проанализировать существующую всячину;
\item спроектировать свою, новую всячину;
\item изготовить всякую всячину;
\item проверить её работоспособность.
\end{itemize}

Вот так-то. А этот абзац вставлен для визуальной оценки отступа от перечня до следующего абзаца.